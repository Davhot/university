\begin{thebibliography}{}

\bibitem{compf}
\link{http://edu1.msiu.ru/f/7275-material\_ici\_toc.zip/index.html}~---
Описание проекта «Стековый копилятор формул».

\bibitem{convex}
\link{http://edu1.msiu.ru/f/7561-material\_ici\_toc.zip/index.html},
\link{http://edu1.msiu.ru/f/7591-material\_ici\_toc.zip/index.html}~---
Описание проекта «Выпуклая оболочка».

\bibitem{polyedr}
\link{http://edu1.msiu.ru/f/7780-material\_ici\_toc.zip/index.html},
\link{http://edu1.msiu.ru/f/7811-material\_ici\_toc.zip/index.html},
\link{http://edu1.msiu.ru/f/7863-material\_ici\_toc.zip/index.html}~---
Описание проекта «Изображение проекции полиэдра».

\bibitem{ruby}
\link{http://ru.wikipedia.org/wiki/Ruby}~---
Википедия (свободная энциклопедия) о языке Ruby.

\bibitem{rlatex}
С.М. Львовский.
{\em Набор и вёрстка в системе \LaTeX, 3-е изд., испр. и доп.}~---
М., МЦНМО, 2003. Доступны исходные тексты этой книги.

\bibitem{texbook}
D.~E.~Knuth. {\em The \TeX{}book.}~---
Addison-Wesley, 1984. Русский перевод:
Дональд~Е.~Кнут.
{\em Все про \TeX.}~--- Протвино, РД\TeX, 1993.

\end{thebibliography}

\section{Введение}
В данной курсовой работе рассматриваются модификации трёх эталонных про-
ектов «Стековый копилятор формул», «Выпуклая оболочка» и 
«Изображение проекции полиэдра», реализованных на объектно-ориентированном 
языке программирования высокого уровня Ruby.

Проект «Стековый копилятор формул»\cite{compf} представляет собой 
программную реализацию некоторой функции $\tau$, действующей из 
множества цепочек одного языка $L_1$ (в рассматриваемом случае 
это язык арифметических формул) в множество цепочек 
другого $L_2$ (язык программ стекового калькулятора) таким образом, 
что $\forall\omega\;\in\;L_1$ семантика 
цепочек $\omega$ и $\tau(\omega)\;\in\;L_2$ совпадает. 
Говоря другими словами, компилятор реализует перевод 
с одного языка на другой с сохранением смысла.

Проект «Выпуклая оболочка»\cite{convex} решает задачу индуктивного 
перевычисления выпуклой оболочки последовательно поступающих точек плоскости 
и таких её характеристик, как периметр и площадь.

Проект «Изображение проекции полиэдра»\cite{polyedr} выполняет построение изображения 
полиэдра с учётом удаления невидимых линий.

Целями работы являются:
\begin{itemize}
\item Проект «Стековый копилятор формул» требуется модифицировать так, 
чтобы вычислялись значения выражений, содержащих только записанные в 
десятичной системе счисления натуральные числа, абсолютная величина 
которых не превосходит 3999.
\item В проект «Выпуклая оболочка» добавить вычисление мощности 
множества точек пересечения границы выпуклой оболочки 
с замкнутым единичным кругом с центром в начале координат.
\item Модифицировать проект «Изображение проекции полиэдра» таким образом, 
чтобы определялась и печаталась следующая характеристика полиэдра: 
сумма длин проекций полностью невидимых рёбер, центр которых находится 
на расстоянии строго меньше единицы от плоскости $x=2$.
\end{itemize}

Для того чтобы выполнить полученные задания, необходимо было изучить 
особенности языка Ruby, подробно разобрать каждый эталонный проект и 
применить полученные знания в области информатики, 
компьютерной математики и аналитической геометрии на плоскости и 
в пространстве.
